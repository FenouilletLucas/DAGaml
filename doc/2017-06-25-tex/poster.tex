\documentclass{beamer}
\usepackage[english]{babel}
\usepackage[utf8]{inputenc}
\usepackage{times}
\usepackage{amsmath,amsthm, amssymb, latexsym}
\boldmath

%\usetheme{Berlin}
\usetheme{Copenhagen}
 
%\usetheme{Sharelatex}
\usepackage[orientation=portrait,size=a0,scale=1.4,debug]{beamerposter}
 
\title[GroBdd : A Generalized Reduction of Ordered Binary Decision Diagram]{A Generalized Reduction of Ordered Binary Decision Diagram\\
  {\small Internship supervised by Rolf Drechsler (DFKI, Bremen, Germany)}}
\author[Joan Thibault, joan.thibault@ens-rennes.fr, +33.6.21.70.56.71]{Joan Thibault}
\date{July 3, 2017}
 
%\logo{\includegraphics[height=7.5cm]{SharelatexLogo}}

\begin{document}
\begin{frame}{} 
\maketitle
\vfill
\begin{block}{\large Introduction to Reduced Ordered Binary Decision Diagrams (ROBDDs)}
\begin{block}{Definition}
ROBDDs are a common and efficient way of canonically representing Boolean functions as Directed Acyclic Graphs.
They have applications in various fields, such as logic synthesis, artificial intelligence and combinatorics. 
They allow to quickly perform various operations such as : equality (\textbf{O(1)}), finding a solution (\textbf{O(\#variable)}), evaluation (\textbf{O(\#variable)}).
ROBDDs can be combined using Boolean operators (OR, AND, XOR) and quantified in quadratic time (\textbf{O(\#node)}).
\end{block}
\begin{block}{Issues}
However, efficiently manipulating ROBDDs requires a lot of memory.
In order to reduce their size, several variations have been introduced.
\end{block}
\begin{block}{}
\end{block}
\end{block}.
 
    \vfill
    \begin{columns}[t]
      \begin{column}{.40\linewidth}
        \begin{block}{Introduction}
          \begin{itemize}
          \item some items
          \item some items
          ...
          \end{itemize}
        \end{block}
      \end{column}
      \begin{column}{.40\linewidth}
        \begin{block}{Introduction}
          \begin{itemize}
          \item some items and $\alpha=\gamma, \sum_{i}$
          ...
          \end{itemize}
          $$\alpha=\gamma, \sum_{i}$$
        \end{block}
        ...
 
      \end{column}
    \end{columns}
    
    
  \end{frame}
\end{document}

