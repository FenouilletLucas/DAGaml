\documentclass[a4paper,10pt]{article}
%\usepackage{fullpage}
\usepackage[T1]{fontenc}
\usepackage[utf8]{inputenc}
\usepackage{lmodern}
\usepackage[english]{babel}
\usepackage{listings}
\usepackage{amsmath}
\usepackage{amssymb}
%\usepackage{mathabx}
\usepackage{listings}
\usepackage{graphicx}
\usepackage{hyperref}

\usepackage{cite}

\lstset{breaklines=true,
  mathescape=true,
	%language=caml,
	numbers=left,
  numberstyle=\tiny \bf, %\color{blue},
  %stepnumber=2,
  numbersep=10pt,
  %firstnumber=11,
  numberfirstline=true
	}

\title{A Generalized Reduction of Ordered Binary Decision Diagram}
\author{Joan Thibault}

\newcommand{\includeframe}[4]{\makebox[#2\linewidth]{\includegraphics[page=#1,width=#2\linewidth,trim=0cm 0cm 0cm 0cm,clip=true,#3]{#4}}}

\newcommand{\shannon}[3]{#1 \longrightarrow_S #2, #3}
\newcommand{\N}{\mathbb{N}}%{\{0, 1\}}
\newcommand{\B}{\mathbb{B}}%{\{0, 1\}}
\newcommand{\F}{\mathbb{F}}%{\{0, 1\}}
\newcommand{\Y}{\mathbb{Y}}%{\{0, 1\}}
\newcommand{\I}{\mathbb{I}}%{\{0, 1\}}
\newcommand{\Ynode}{\Y\mathtt{-node}}

\begin{document}

\section{GroBdd}

\subsection{Initial definition of GroBdd}

A GroBdd is very similar to an actual ROBDD, the two main differences being that (1) it represents a vector of Boolean functions with a finite number of variables possibly of different arity and (2) on every edge their is a transformation (the "output inverter" is an example of such transformations").

A Reduced Ordered Binary Decision Diagram is a directed acyclic graph $(V\cup T, \Psi \cup E)$ representing a vector of Boolean functions $F=(f_1, ..., f_k)$.
Nodes are partitioned into two sets : the set of internal nodes $V$ and the set of terminal nodes $T$.
Every internal node $v\in V$ has two outgoing edges respectively denoted $if0$ and $if1$.
Every internal node $v\in V$ has a field $index$ which represent a unique identifier associated to each node.
Arcs are partitioned into two sets : the set of root arcs $\Psi$ and the set of internal arcs $E$.
There is exactly $k$ root arcs, a root arc is denoted $\Psi_i$ with $0\leq i < k$, informally, $\Psi_i$ is the root of the GroBdd representing $f_i$.
Every arc has a transformation descriptor field $\gamma$ and a destination node denoted $node$.

%add beautiful draw (with \psi arcs)$
We denote $\rho(\gamma) : \F_n \longrightarrow F_m$ the semantic interpretation of the transformation descriptor $\gamma$.
We define $\phi(node)$ the semantic of the node $node$ and $\psi(arc)$ the semantic of the arc $arc$ as follow:\begin{itemize}
\item $\forall i, f_i = \psi(\Psi_i)$
\item $\forall arc \in \Psi \cup E, \psi(arc) = \rho(arc.\gamma)(\phi(arc.node))$
\item $\forall node \in V, \phi(node) = \psi(node.if0) \star \psi(node.if1)$
\end{itemize}
We assume the function $\phi$ defined on all terminals $T$ (we always assume, all terminal to have a different interpretation through $\phi$).

\subsection{Transformation Descriptor Set (TDS)}

We denote $\Y$ the set of all transformation descriptor.
We assumme defined $\rho$ the semantical interpretation of transformation descriptors, $\forall \gamma, \exists n, m \in\N, \rho(\gamma) \in \F_n \rightarrow \F_m$ (with $n\leq m$).
In this section, we make the list of properties that the TDS must ensure to be correct.

\subsubsection{Canonical}
\[\forall \gamma, \gamma' \in \Y, \left( \gamma = \gamma' \right) \Leftrightarrow \left( \rho(\gamma) = \rho(\gamma') \right) \]
We denote $\Y_{n, m} = \{ \gamma \in \Y ~|~ \rho(\gamma) \in \F_n \longrightarrow \F_m \}$, and $\Y_{n,*} = \bigcup_{m\leq n} \Y_{n, m}$ and $\Y_{*, m} = \bigcup_{n \geq m} \Y_{n, m}$.

\subsubsection{Separable}
\[\forall \gamma \in \Y_{n, m}, \exists! \vartriangle_\gamma \in \B^m \rightarrow \B^n, \triangledown_\gamma \in \B^m \rightarrow \B \rightarrow \B,\]
\[\forall f\in \F_n, \forall x \in \B^m, \rho(\gamma)(f)(x) = \triangledown(x, f(\vartriangle x))\]
This constraint allows any transformation to be represented as circuit which can be wrapped around the function.
This constraint enforces transformations to be local.
The function $\vartriangle$ is called the pre-process, and the function $\triangledown$ is called the post-process.

\subsubsection{Composable (definition of \texttt{C})}
\[\forall \gamma \in \Y_{n, m}, \gamma' \in \Y_{m, l}, \exists \gamma'' \in \Y_{n, l}, \rho(\gamma') \circ 
\rho(\gamma) = \rho(\gamma'')\]
This constraint enforces $\Y_{n, n}$ to be stable by composition.
Furthermore, it exists an algorithm $\mathtt{C} : \Y_{n, m} \rightarrow \Y_{m, l} \rightarrow \Y_{n, l}$, such that:
\[\forall \gamma \in \Y_{n, m}, \gamma' \in \Y_{m, l}, \rho(\gamma') \circ \rho(\gamma) = \rho(\mathtt{C}(\gamma, \gamma'))\]
For convenience, we denote $\gamma' \circ \gamma = \mathtt{C}(\gamma, \gamma')$.

\subsubsection{Decomposable (definition of \texttt{A} and \texttt{S})}

For all $n\in\N$, we define $A_n = \Y_{n, n}$ the set of asymmetric transformations.
For all $n, m\in\N$, it exists $S_{n, m} \subset \Y_{n, m}$ a set of transformations such that $\forall \gamma \in \Y_{n, m}, \exists a \in A_m, \exists! s \in S_{n, m},  \gamma = a \circ s$.
The set $S_{n, m}$ is called the set of symmetric transformation.
%Consequence : $S_{n, n} = \{ Id_n = \F_n \rightarrow \F_n \}$.

\paragraph{definition : S-free\\}
A Boolean function $f\in\F_m$ is said S-free, iff
\[\forall s \in S_{n, m}, \forall g \in \F_n, f = \rho(s)(g) \Rightarrow \rho(s) = Id\]

\paragraph{constraint : S-uniqueness\\}
\[\forall f\in\F_n, f'\in\F_{n'}, s\in S_{n, m}, s'\in S_{n', m}, \rho(s)(f) = \rho(s')(f') \Rightarrow (s = s') \land (f = f')\]

\paragraph{definition : A-equivalent\\}
Two Boolean functions $f, g\in\F_n$ are said A-equivalent, iff $\exists a\in A_n, f = \rho(a)(g)$.
This relation is an equivalence relation (i.e. reflexive, symmetric, transitive) denote $\sim_A$.

\paragraph{definition : A-invariant free\\}
A Boolean function $f\in\F_n$ is said A-invariant free, iff $\forall a, a'\in A_n, \rho(a)(f) \neq \rho(a')(f)$.

\paragraph{constraint : S-free implies A-invariant free\\}
For all function $f$, if $f$ is S-free, then $f$ is A-invariant free.

\paragraph{definition : A-reduced set of function\\}
Let $X = \{x_1, x_2, \dots, x_n\}$ be a set of Boolean functions.
The set $X$ is said A-reduced iff $\forall x_i, x_j \in X, (x_i \sim_A x_j) \Rightarrow (x_i = x_j)$.

\paragraph{definition : $\I_n$\\}
For all $n\in\N$, we denote $\I_n$ the set of identifiers corresponding to node representing functions of arity $n$.

\paragraph{definition : $\Ynode$\\}
A $\Ynode_{l, m, n}$ is a quadruple $(\gamma_0, I_0, \gamma_1, I_1) \in \Y_{l, n} \times \I_l \times \Y_{m, n} \times \I_m$. Let $v$ be a $\Ynode$, we denote $v.\gamma_0$ (respectively $v.I_0$, $v.\gamma_1$ and $v.I_1$) the first (respectively second, third and fourth) component of $v$.
\begin{itemize}
\item For all $\Ynode$ $v$, we always assume that functions $\phi(v.I_0)$ and $\phi(v.I_1)$ are \texttt{S}-free.
\item We always assume the set $X = \{\phi(v.I_0) ~|~ v \in\Ynode\} \cup \{\phi(v.I_1) ~|~ v\in\Ynode\}$ to be \texttt{A-reduced}
\end{itemize}
We extend the definition of $\phi$, with : for all $\Ynode$ $v$, $\phi(v) = \rho(v.\gamma_0)(\phi(v.I_0)) \star \rho(v.\gamma_1)(\phi(v.I_1))$.

\paragraph{constraint : terminal nodes are S-free and A-reduced (definition of $\mathtt{E}_0$)}
For all terminal node $t \in T$, $t$ is S-free.
Furthermore, the set $\{\phi(t) ~|~ t\in T\}$ is A-reduced.
Moreover, we define the function $\mathtt{E}_0$ such $\forall b\in\F_0, \exists \gamma\in\Y_{0, 0}, \exists t\in T, \mathtt{E}_0(b) = \{\gamma = \gamma, node = I_t\}$ and $\psi(\mathtt{E}_0(b)) = b$.


\subsubsection{Buildable (definition of \texttt{B})}

Let $X$ be a function, we denote $I_X$ the identifier of an hypothetical node whose semantic interpretation is $X$.

We define \texttt{B} an algorithm over $\Y$ which respect the signature:
\begin{lstlisting}
$\mathtt{B} : \Y_{n_0, m} \times \I_{n_0} \longrightarrow \Y_{n_1, m} \times \I_{n_1} \longrightarrow$
  | ConsNode $\Y_{n', m} \times (\Y_{n_x, n'} \times \I_{n_x}) \times (\Y_{n_y, n'} \times \I_{n_y})$
  | Merge $\Y_{n_z, m} \times \I_{n_z}$
\end{lstlisting}
(with $x, y, z \in \{0, 1\}$) \\
Furthermore, for all $(\gamma_g, I_g, \gamma_h, I_h) \in \Y_{n_0, m} \times \I_{n_0} \times \Y_{n_1, m} \times \I_{n_1}$, 
\[ \texttt{B}(\gamma_g, I_g, \gamma_h, I_h) = \texttt{ConsNode} (\gamma, (\gamma', I_X), (\gamma'', I_Y)) \Rightarrow f = \rho\left(\gamma\right) \left(\rho\left(\gamma'\right)(X) \star \rho\left(\gamma''\right)(Y)\right)\]
\[ \texttt{B}(\gamma_g, I_g, \gamma_h, I_h) = \texttt{Merge} (\gamma''', I_Z) \Rightarrow f = \rho(\gamma''')(Z) \]
(with $X, Y, Z \in\{g, h\}$ and $f = \rho(\gamma_g)(g) \star \rho(\gamma_h)(y)$)

\paragraph{definition : a node is \texttt{B}-stable\\}
Let $G$ be a GroBdd, we denote $v$ an internal node of $G$.
We denote $\gamma_0 = v.if0.\gamma, I_0 = v.if0.node, \gamma_1 = v.if1.\gamma$ and $I_1 = v.if1.node$.
The node $v$ is said B-stable iff $\mathtt{B}(\gamma_0, I_0, \gamma_1, I_1) = \mathtt{ConsNode}(Id, (\gamma_0, I_0), (\gamma_1, I_1))$.

\paragraph{constraint : \texttt{B} is \texttt{B}-stable\\}
\[\forall \gamma_f, I_f, \gamma_g, I_g, \mathtt{B}(\gamma_f, I_f, \gamma_g, I_g) = \mathtt{ConsNode}(\gamma, (\gamma_0, I_0), (\gamma_1, I_1))\]
\[\Rightarrow \mathtt{B}(\gamma_0, I_0, \gamma_1, I_1) = \mathtt{ConsNode}(Id, (\gamma_0, I_0), (\gamma_1, I_1))\]
Informally, when B returns a node, this node is \texttt{B}-stable.

\paragraph{constraint : $\Ynode$ are \texttt{B}-stable}
\begin{enumerate}
\item We assume all $\Ynode$ $v$ to be \texttt{B}-stable
\[\mathtt{B}(v.\gamma_0, v.I_0, v.\gamma_1, v.I_1) = \mathtt{ConsNode}(Id, (v.\gamma_0, v.I_0), (v.\gamma_1, v.I_1))\]
\end{enumerate}

\paragraph{constraint : B is S-free preserving\\}
The algorithm \texttt{B} is said S-free preserving iff for all $\Ynode$ $v$, $\phi(v)$ is S-free.

\paragraph{constraint : B is A-reduction preserving\\}
The algorithm \texttt{B} is said A-reduction preserving iff
\[\forall v, w \in\Ynode, \phi(v) \sim_A \phi(w) \Rightarrow v = w\]

\subsection{Reduction Rules}
We define a GroBdd model as the triple $(\Y, \mathtt{C}, \mathtt{B})$.
A valid model must satisfy all the previously mentioned constraints.

In addition to the previous constraints, we define two reduction rules:\begin{enumerate}
\item The syntactical reduction : all sub-graphs are different up to graph-isomorphism (i.e. all identical sub-graphs are merged)
\item The local semantic reduction : all internal node $v\in V$ is \texttt{B}-stable.
\item All node has at least one incoming arc.
A GroBdd is said reduced if it satisfies the reduction rules.
\end{enumerate}
In this section we prove:\begin{enumerate}
\item For all vector of Boolean function $F$ it exists a reduced GroBdd $G$ representing it.
\item A reduced GroBdd $G$ is semi-canonical, defined as :\begin{enumerate}
\item For all node $v\in V$, $\phi(v)$ is S-free.
\item The set $X = \{\phi(v_1), \dots, \phi(v_N)\}$ representing the set of the semantic interpretation of the set of internal nodes $V$ is A-reduced.
\item $\forall (\gamma, I, \gamma', I') \in (\Y_{n, m} \times \I_n)^2, \psi(\{\gamma = \gamma, node = I\}) = \psi(\{\gamma = \gamma', node = I'\}) \Rightarrow (\gamma, I) = (\gamma', I')$ (with $I$ and $I'$ being indexes of nodes in $G$).
\end{enumerate}
\item Between two reduced GroBdd $G$ and $G'$ representing the same vector of Boolean functions $F$, it exists a one-to-one mapping $\sigma : V \longrightarrow V'$ such that $\forall v, v' \in V \times V', \sigma(v) = v' \Rightarrow (\exists a \in A_{*}, \phi(v) = \rho(a)(\phi(v'))$.
\end{enumerate}

\subsubsection{Additional procedures}

\paragraph{The \texttt{Cons} procedure\\}
We define the \texttt{Cons} procedure as follow.
Let $G$ be a GroBdd and $F$ be its vector of root arcs of size $k$.\\
Let $i$ and $j$ be indexes of this vector.
We denote $\gamma_i = \Psi_i.\gamma$, $\gamma_j = \Psi_j.\gamma$ and $I_i = \Psi_i.node$, $I_j = \Psi_j.node$.
\begin{itemize}
\item If \texttt{B}($\gamma_g$, $I_g$, $\gamma_h$, $I_h$) = \texttt{ConsNode} ($\gamma$, ($\gamma'$, $I_X$), ($\gamma''$, $I_Y$)). We define the node $N = \{if0 = \{\gamma = \gamma', node = I_X\}, if1 = \{\gamma = \gamma'', nodes = I_Y\}\}$.\begin{itemize}
\item If the node $N$ already exists in $G$, we retrieve its identifier $I$, we define $G'$ as a copy of $G$ with a new root arc $\Psi_k = \{\gamma = \gamma, node = I\}$
\item Otherwise, we define $G'$ as a copy of $G$ with add $N$ to the list of nodes of $G'$ and create it a new identifier $I$, we add a new root arc $\Psi_k = \{\gamma = \gamma, node = I\}$.
\end{itemize}
\item Otherwise, \texttt{B}($\gamma_g$, $I_g$, $\gamma_h$, $I_h$) = \texttt{Merge} ($\gamma$, $I_Z$). We define $G'$ as a copy of $G$ with a new root arc $\Psi_k = \{\gamma = \gamma''', node = I_Z\}$.
\end{itemize}

\paragraph{The \texttt{Remove} procedure\\}
We define the \texttt{Remove} procedure as follow.
Let $G$ be a GroBdd and $F$ be its vector of root arc of size $k$.\\
Let $i$ be an index of this vector
We define $G'$ as a copy of $G$ with its vector of root arc $F' = (\Psi_1, \dots, \Psi_{i-1}, \Psi_{i+1}, \dots, \Psi_n)$.
In an iterative process, we remove nodes which have no incoming arc, until all nodes have at least one incoming arc (in order to satisfy the third reduction rule).
As $G'$ is a subset of $G$, thus, $G'$ satisfies the first and second reduction rule.

\paragraph{The \texttt{Reduction} procedure\\}
Let $G$ be a GroBdd and $F$ be its vector of root arcs $F$.
Using an iterative process, we remove nodes which have no incoming arc, until all nodes have at least one incoming arc (satisfying the third reduction rules).
The \texttt{Reduction} procedure consist in going through the GroBdd $G$ (starting with nodes with the smallest depth), applying the \texttt{Cons} procedure on each node creating a new GroBdd $G'$.
The GroBdd $G'$ is by construction equivalent to $G$, however, the GroBdd $G'$ satisfies all three reduction rules.

\paragraph{The \texttt{Merge} procedure\\}
We define the \texttt{Merge} procedure as follow.
Let $G$ and $G'$ be two GroBdds (based on the same GroBdd model).
We define $G''$ a GroBdd which is the union of both GroBdd.
We define $F'' = (\Psi_1, \dots, \Psi_n, \Psi'_1, \dots, \Psi'_{n'})$.
Due to possible conflicts on identifiers, we re-generate identifiers of the nodes in $G''$.
Finally, we apply the \texttt{Reduction} procedure on $G''$.

\subsubsection{Existence}
We inductively define the procedure \texttt{E} with:\begin{itemize}
\item $\forall b\in\F_0, \texttt{E}(b) = \texttt{E}_0(b)$
\item
Let $f$ be a Boolean function of arity $n$ (with $n \geq 1$).
Let $G$ be a GroBdd representing functions $f_0$ (the negative restriction of $f$ according to its first variable) and $f_1$ (the positive restriction of $f$ according to its first variable.) by using the procedure \texttt{E} on $f_0$ and $f_1$.
Let $G'$ be the output of the \texttt{Cons} procedure on $G$, in order to create $f = f_0 \star f_1$ (expansion theorem).\\
$E(f) = G'$
The GroBdd $G'$ satisfies the reduction rules:\begin{itemize}
\item If no node is created, the proof is straightforward.
\item If a node is created, this node is syntactically unique by definition of \texttt{Cons} and is \texttt{B}-stable  (as \texttt{B} is \texttt{B}-stable)
\end{itemize}
\end{itemize}

By construction, the procedure \texttt{E} (generalized to accept a vector of function as input) returns a GroBdd satisfying the reduction rules.

\subsubsection{Semi-Canonical}

Let $G$ be a reduced GroBdd.

\paragraph{S-free and A-reduced}
For all node $v\in V$, we denote $h(v) = max(h(v.if0.node), h(v.if1.node))$ with $\forall t\in T, h(t) = 0$.
For all $n\in\N$, we define $V_n = \{v\in V ~|~ h(v) \leq n\}$
For all $n\in\N$, we define the recurrence hypothesis $H(n)$ :\begin{itemize}
\item For all $v\in V_n$, $\phi(v)$ is S-free.
\item The set of Boolean function $X = \{\phi(v) ~|~ v\in V_n\}$ is A-reduced.
\end{itemize}

\subparagraph{Initialization}
We prove $H(0)$ using the constraints that (1) terminals are S-free and (2) the set of terminal nodes is A-reduced.

\subparagraph{Induction}
Let $n\in\N$, we assume $\forall k\leq n H(k)$.
Let $v$ be a node of depth $n+1$, thus the depth of $v.if0.node$ and $v.if1.node$ is lower than $n$ (we can apply the recurrence hypothesis).
Therefore, the quadruple $\bar{v} = (v.if0.\gamma, v.if0.node, v.if1.\gamma, v.if1.node)$ is a $\Ynode$.
Thus, using the constraints that \texttt{B} is S-free preserving, we prove that $\phi(v) = \phi(\bar{v})$ is S-free.
Let $v'$ be a node of depth $k\leq n+1$, we can prove that the quadruple $\bar{v'} = (v'.if0.\gamma, v'.if0.node, v'.if1.\gamma, v'.if1.node)$ is a $\Ynode$.
Therefore, we can use the constraint that \texttt{B} is A-reduction preserving to prove that $\phi(\bar{v}) \sim_A \phi(\bar{v'}) \Rightarrow \phi(\bar{v}) = \phi(\bar{v'})$
However, $\phi(v) = \phi(\bar{v})$ and $\phi(v') = \phi(\bar{v'})$
Thus, $\forall v, w \in V_{n+1}, \phi(v) \sim_A \phi(v') \Rightarrow \phi(v) = \phi(v')$.
Thus, the set of Boolean function $X = \{\phi(v) ~|~ v\in V_{n+1}\}$ is A-reduced.
Therefore $(land_{k\leq n} H(k) \Rightarrow H(n+1)$.

Using the strong recurrence theorem, we prove that $\forall n\in\N, H(n)$.
Therefore proving properties (2.a) "all nodes are S-free" and (2.b) "the set $X = \{\phi(v_1), \dots, \phi(v_N)\}$ is A-reduced".

\paragraph{Semantic Reduction}
We prove the property "$\forall (\gamma, I, \gamma', I') \in (\Y_{n, m} \times \I_n)^2, \psi(\{\gamma = \gamma, node = I\}) = \psi(\{\gamma = \gamma', node = I'\}) \Rightarrow (\gamma, I) = (\gamma', I')$ (with $I$ and $I'$ being indexes of nodes in $G$)" by induction on $n\in\N$ the arity of $f = \psi(\{\gamma = \gamma, node = I\})$.

\subparagraph{Initialization}
The induction property holds for $n = 0$:\\
Let $f\in\F_0$, we assume it exists a quadruple $(\gamma, I, \gamma', I') \in (\Y_{n, m} \times \I_n)^2$ such that $f = \psi(\{\gamma = \gamma, node = I\}) = \psi(\{\gamma = \gamma', node = I'\})$.
We decompose $\gamma$ and $\gamma'$ to their symmetric and asymmetric components: $\gamma = s \circ a$ and $\gamma' = s' \circ a'$.
Using the S-uniqueness constraint, on $f = \rho(s)(\rho(a)(\phi(I))) = \rho(s')(\rho(a')(\phi(I')))$, we have $s = s'$ and $\rho(a)(\phi(I)) = \rho(a')(\phi(I'))$.
Therefore, $\phi(I) \sim_A \phi(I')$, however, we proved that the set of the semantic interpretations of the nodes is A-reduced, thus $\phi(I) = \phi(I')$
However, the only node representing function of arity 0 are terminal nodes, therefore $I = I'$.

\subparagraph{Induction}
Let $k\in\N$, we assume the induction property holds for all $n\leq k$, let prove it holds for $n = k+1$.
Let $f$ be a Boolean function, we assume it exists a quadruple $(\gamma, I, \gamma', I') \in (\Y_{n, m} \times \I_n)^2$ such that $f = \psi(\{\gamma = \gamma, node = I\}) = \psi(\{\gamma = \gamma', node = I'\})$.
We decompose $\gamma$ and $\gamma'$ to their symmetric and asymmetric components: $\gamma = s \circ a$ and $\gamma' = s' \circ a'$.
Using the S-uniqueness constraint, on $f = \rho(s)(\rho(a)(\phi(I))) = \rho(s')(\rho(a')(\phi(I')))$, we have $s = s'$ and $\rho(a)(\phi(I)) = \rho(a')(\phi(I'))$.
Therefore, $\phi(I) \sim_A \phi(I')$, however, we proved that the set of the semantic interpretations of the nodes is A-reduced, thus $\phi(I) = \phi(I')$
Using the induction hypothesis (as $\phi(I)$ as an arity strictly smaller than $k+1$, thus smaller than $k$), we deduce that $I = I'$.

Therefore, applying the strong induction theorem, we prove the property (2.c).

\subsubsection{Canonical modulo graph-isomorphism and A-equivalence}
In order to prove that "Between two reduced GroBdd $G$ and $G'$ representing the same vector of Boolean functions $F$, it exists a one-to-one mapping $\sigma : V \longrightarrow V'$ such that $\forall v, v' \in V \times V', \sigma(v) = v' \Rightarrow (\exists a \in A_{*}, \phi(v) = \rho(a)(\phi(v'))$", we start by (1) proving that within the \texttt{Merge} procedure, each node is replace by an A-equivalent one then (2) using the \texttt{Merge} procedure without re-generating the identifiers of the nodes of $G$ and using the property (2.c), we prove that the nodes of $G'$ collapse on the nodes of $G$ during the \texttt{Reduction} procedure, providing a one-to-one mapping.
Details of this proof are cumbersome and not of great interest


\end{document}
